%%%%%%%%%%%%%%%%%%%%%% Simple LaTeX CV flate %%%%%%%%%%%%%%%%%%%%%%%%

%% NOTE: If you find that it says                                     %%
%%                                                                    %%
%%                           1 of ??                                  %%
%%                                                                    %%
%% at the bottom of your first page, this means that the AUX file     %%
%% was not available when you ran LaTeX on this source. Simply RERUN  %% 
%% LaTeX to get the ``??'' replaced with the number of the last page  %% 
%% of the document. The AUX file will be generated on the first run   %%
%% of LaTeX and used on the second run to fill in all of the          %%
%% references.                                                        %%
%%%%%%%%%%%%%%%%%%%%%%%%%%%%%%%%%%%%%%%%%%%%%%%%%%%%%%%%%%%%%%%%%%%%%%%%

%%%%%%%%%%%%%%%%%%%%%%%%%%%% Document Setup %%%%%%%%%%%%%%%%%%%%%%%%%%%%

% Don't like 10pt? Try 11pt or 12pt
\documentclass[12pt]{article}

% This is a helpful package that puts math inside length specifications
\usepackage{calc}
\usepackage{graphicx, amssymb, amsbsy, amsmath, MnSymbol}
\usepackage{helvet}
\renewcommand{\familydefault}{\sfdefault}
% \usepackage{MinionPro}
\usepackage[normalem]{ulem}
\usepackage{tabularx}
\usepackage{url}


% Layout: Puts the section titles on left side of page
\reversemarginpar

%
%         PAPER SIZE, PAGE NUMBER, AND DOCUMENT LAYOUT NOTES:
%
% The next \usepackage line changes the layout for CV style section
% headings as marginal notes. It also sets up the paper size as either
% letter or A4. By default, letter was used. If A4 paper is desired,
% comment out the letterpaper lines and uncomment the a4paper lines.
%
% As you can see, the margin widths and section title widths can be
% easily adjusted.
%
% ALSO: Notice that the includefoot option can be commented OUT in order
% to put the PAGE NUMBER *IN* the bottom margin. This will make the
% effective text area larger.
%
% IF YOU WISH TO REMOVE THE ``of LASTPAGE'' next to each page number,
% see the note about the +LP and -LP lines below. Comment out the +LP
% and uncomment the -LP.
%
% IF YOU WISH TO REMOVE PAGE NUMBERS, be sure that the includefoot line
% is uncommented and ALSO uncomment the \pagestyle{empty} a few lines
% below.
%

%% Use these lines for letter-sized paper
%\usepackage[paper=letterpaper,
%            %includefoot, % Uncomment to put page number above margin
%            marginparwidth=1.2in,     % Length of section titles
%            marginparsep=.05in,       % Space between titles and text
%            margin=1in,               % 1 inch margins
%            includemp]{geometry}

%% Use these lines for A4-sized paper
\usepackage[paper=letterpaper,
            %includefoot, % Uncomment to put page number above margin
            marginparwidth=30.5mm,    % Length of section titles
            marginparsep=1.5mm,       % Space between titles and text
            margin=15mm,              % 25mm margins
            includemp]{geometry}

%% More layout: Get rid of indenting throughout entire document
\setlength{\parindent}{0in}

%% This gives us fun enumeration environments. compactenum will be nice.
\usepackage{paralist}
\usepackage{tabto}

%% Reference the last page in the page number
%
% NOTE: comment the +LP line and uncomment the -LP line to have page
%       numbers without the ``of ##'' last page reference)
%
% NOTE: uncomment the \pagestyle{empty} line to get rid of all page
%       numbers (make sure includefoot is commented out above)
%
\usepackage{fancyhdr,lastpage}
\pagestyle{fancy}
%\pagestyle{empty}      % Uncomment this to get rid of page numbers
\fancyhf{}\renewcommand{\headrulewidth}{0pt}
\fancyfootoffset{\marginparsep+\marginparwidth}
\newlength{\footpageshift}
\setlength{\footpageshift}
          {0.5\textwidth+0.5\marginparsep+0.5\marginparwidth-2in}
\lfoot{\hspace{\footpageshift}%
       \parbox{4in}{\, \hfill %
                    \arabic{page} of \protect\pageref*{LastPage} % +LP
%                    \arabic{page}                               % -LP
                    \hfill \,}}

% Finally, give us PDF bookmarks
\usepackage{color,hyperref}
\definecolor{darkblue}{rgb}{0.0,0.0,0.3}
\hypersetup{colorlinks,breaklinks,
            linkcolor=darkblue,urlcolor=darkblue,
            anchorcolor=darkblue,citecolor=darkblue}

%%%%%%%%%%%%%%%%%%%%%%%% End Document Setup %%%%%%%%%%%%%%%%%%%%%%%%%%%%


%%%%%%%%%%%%%%%%%%%%%%%%%%% Helper Commands %%%%%%%%%%%%%%%%%%%%%%%%%%%%

% The title (name) with a horizontal rule under it
%
% Usage: \makeheading{name}
%
% Place at top of document. It should be the first thing.
\newcommand{\makeheading}[1]%
        {\hspace*{-\marginparsep minus \marginparwidth}%
         \begin{minipage}[t]{\textwidth+\marginparwidth+\marginparsep}%
                {\large \bfseries #1}\\[-0.15\baselineskip]%
                 \rule{\columnwidth}{1pt}%
         \end{minipage}}

% The section headings
%
% Usage: \section{section name}
%
% Follow this section IMMEDIATELY with the first line of the section
% text. Do not put whitespace in between. That is, do this:
%
%       \section{My Information}
%       Here is my information.
%
% and NOT this:
%
%       \section{My Information}
%
%       Here is my information.
%
% Otherwise the top of the section header will not line up with the top
% of the section. Of course, using a single comment character (%) on
% empty lines allows for the function of the first example with the
% readability of the second example.
\renewcommand{\section}[2]%
        {\pagebreak[2]\vspace{1.3\baselineskip}%
         \phantomsection\addcontentsline{toc}{section}{#1}%
         \hspace{0in}%
         \marginpar{
         \raggedright \scshape #1}#2}

% An itemize-style list with lots of space between items
\newenvironment{outerlist}[1][\textbullet]%
        {\begin{enumerate}[#1]}{\end{enumerate}%
         \vspace{-6mm}}%\baselineskip}}

% An itemize-style list with little space between items
\newenvironment{innerlist}[1][\textbullet]%
        {\begin{compactenum}[#1]}{\end{compactenum}}

% To add some paragraph space between lines.
% This also tells LaTeX to preferably break a page on one of these gaps
% if there is a needed pagebreak nearby.
\newcommand{\blankline}{\quad\pagebreak[2]}

%%%%%%%%%%%%%%%%%%%%%%%% End Helper Commands %%%%%%%%%%%%%%%%%%%%%%%%%%%

%%%%%%%%%%%%%%%%%%%%%%%%% Begin CV Document %%%%%%%%%%%%%%%%%%%%%%%%%%%%

\begin{document}
\makeheading{{\huge \textbf{Marc Chengliang Zhang}}}

\section{Present Position}
%
% NOTE: Mind where the & separators and \\ breaks are in the following
%       table.
%
% ALSO: \rcollength is the width of the right column of the table 
%       (adjust it to your liking; default is 1.85in).
%
\newlength{\rcollength}\setlength{\rcollength}{2.4in}%
%
\begin{tabular}[t]{@{}p{\textwidth-\rcollength}p{\rcollength}}
	 Ph.D Candidate & \emph{Office}: Room CYT 3007 \\
     Dept. of Computer Science \& Engineering & \emph{Phone}:  +852--6216--2287\\
	 Hong Kong University of Science and Technology  & \emph{Email}: \href{mailto: czhangbn@cse.ust.hk}{czhangbn@cse.ust.hk} \\
     Clear Water Bay, Hong Kong  &  \emph{Web}: \href{https://marcoszh.github.io/}{https://marcoszh.github.io/} \\
\end{tabular}

% \section{Personal Information}
% \newlength{\shortcollength}\setlength{\shortcollength}{4.5in}%
% %
% \begin{tabular}[t]{@{}p{\textwidth-\shortcollength}p{\shortcollength}}
%      \emph{Citizenship} & Chinese \\
%      \emph{Permanent Residency} &  Canadian
% \end{tabular}

\section{Research Interests}
My interests cover \textbf{big data analytics systems} and \textbf{cloud computing}, with a special focus on \textbf{machine learning systems}. I enjoy identifying fundamental system design and performance issues in large-scale ML systems for both training and inference, and searching for general and efficient solutions.

\section{Education}
\textbf{Hong Kong University of Science and Technology}, Hong Kong SAR\\
{\em Department of Computer Science and Engineering}\\[0.4em]
        \mbox{}$\diamondsuit$ \textbf{Ph.D.} Computer Science and Engineering \hfill September 2016 - present\\
        \mbox{}\hspace{6mm}\hspace{6mm}$\diamondsuit$ \emph{Supervisor:} \href{https://www.cse.ust.hk/~weiwa/}{Wei Wang}\\
        \mbox{}\hspace{6mm}\hspace{6mm}$\diamondsuit$ \href{https://cerg1.ugc.edu.hk/hkpfs/index.html}{Hong Kong PhD Fellowship} recipient: prestigious and highly selective fellowship\\[-0.4em]

\textbf{Harbin Institute of Technology}, Harbin, China\\
{\em School of Computer Science and Technology}\\[0.4em]
        \mbox{}$\diamondsuit$ \textbf{B.Eng.} Software Engineering \hfill September 2012 - June 2016\\
        \mbox{}\hspace{6mm}\hspace{6mm}$\diamondsuit$ \emph{Honors:} National Scholarship (Top 2\%), People's Scholarship, Fuji Xerox Scholarship

% % \vspace{-.2in}
% \section{Professional Experience}
% \textbf{Hong Kong University of Science and Technology}, Clear Water Bay, Kowloon, Hong Kong\\
% Department of Computer Science and Engineering\\[-0.6em]

% \emph{Assistant Professor (tenure-track)} \hfill September 2015 -- Present\\

% \textbf{University of Toronto}, Toronto, Ontario, Canada\\
% Department of Electrical and Computer Engineering\\[-0.6em]

% \emph{Research/Teaching Assistant} \hfill September 2010 -- August 2015\\

% \textbf{Microsoft Server and Tools Business}, Shanghai, China\\[-0.6em]

% \emph{Software Development Engineer (intern)}\\
% \mbox{}\hspace{6mm}$\diamondsuit$ Placement and routing algorithms for Microsoft WorkFlow 4.0  \hfill March -- June, 2010\\
% \mbox{}\hspace{6mm}$\diamondsuit$ Cloud identity service development for Microsoft BizTalk  \hfill June -- September, 2008\\

% \textbf{Microsoft Research Asia}, Beijing, China\\[-0.6em]

% \emph{Research Assistant} \hfill March -- June 2007

\section{Publications}

\uline{Chengliang Zhang}, {Minchen Yu}, {Wei Wang}, Feng Yan, ``\href{https://marcoszh.github.io/publication/mark/}{MArk: Exploiting Cloud Services for Cost-Effective, SLO-Aware Machine Learning Inference Serving},'' in the {\em Proceedings of USENIX Annual Technical Conference (ATC'19)}, Renton, WA, July 2018 (20\% acceptance rate).\\


\uline{Chengliang Zhang}, {Huangshi Tian}, {Wei Wang}, Feng Yan, ``\href{https://marcoszh.github.io/publication/spec-sync/}{Stay Fresh: Speculative Synchronization for Fast Distributed Machine Learning},'' in the {\em Proceedings of IEEE International Conference on Distributed Computing Systems (ICDCS'18)}, Vienna, Austria, July 2018 (20\% acceptance rate).\\

\textbf{Preprints}\\

\uline{Chengliang Zhang}, {Minchen Yu}, {Wei Wang}, Feng Yan, ``Towards Cost-Effective and SLO-Aware Machine Learning Inference Serving on Public Cloud,'' to be submitted to {\em IEEE Transactions on Parallel and Distributed Systems}.\\

{Yinghao Yu}, \uline{Chengliang Zhang}, {Wei Wang}, {Jun Zhang}, {Khaled Letaief}, ``Towards Dependency-Aware Cache Management for Data Analytics Applications,'' submitted to {\em IEEE Transactions on Cloud Computing}, currently under review.

\section{Research Experience}
\textbf{Fast Secure Federated Learning System}
\vspace{2pt}
\begin{compactenum}[-]
\item Inter-enterprise federated learning with Homomorphic Encryption
\item Accelerate training and inference of Secure Federated Learning
\item Mitigate encryption and communication overhead
\end{compactenum}
\vspace{5pt}

% I am currently working on accelerating the training and inference process of secure federated learning. We partner with a large commercial bank, and focus on enabling inter-enterprise federated learning with the help of Homomorphic Encryption. Specifically, I am working on how to mitigate the encryption and communication overhead. \\[-0.4em]

\textbf{MArk: ML Serving on Public Cloud}
\vspace{2pt}
\begin{compactenum}[-]
\item Serve machine learning inference on public cloud
\item Cost-effective and SLO-aware
\item Characterization of ML serving and its performance cloud services
\item Combine FaaS and IaaS to reduce over-provisioning
\item Characterization of hardware accelerators like GPU and TPU
\end{compactenum}
\vspace{5pt}
% We aspired to serve machine learning models on public cloud with both SLO compliance and cost-effectiveness. We first characterized ML serving and its performance on diverse cloud services, and then designed MArk based on our insights. MArk uses predictive autoscaling to maintain high utilization, and eliminates over-provisioning by utilizing flexible, yet expensive FaaS. MArk further brings down the cost by adopting ML accelerators judiciously. \\[-0.4em]

\textbf{Speculative Synchronization}
\vspace{2pt}
\begin{compactenum}[-]
\item Distributed data parallel training
\item Relaxed consistency can increase throughput but hurt update quality
\item Re-synchronize if the parameter copy is too stale to produce beneficial updates
\end{compactenum}
% \vspace{5pt}
% Asynchronous parallel can improve distributed ML training's throughput. However, the introduced inconsistency leads to low quality updates. We proposed Speculative Synchronization to exploit the trade-off between training throughput and update quality, which allows a worker to abort the current computation and synchronize, if it is confident that the benefits of fresher parameter outweight the loss of completed computation.


% \section{Teaching Experience}
% Hong Kong University of Science and Technology
% \vspace{2mm}
% \begin{innerlist}\itemsep0.25em
%     \item[$\diamondsuit$] COMP3511: \emph{Operating System} \hfill Spring, 2018, 2019
%     \item[$\diamondsuit$] COMP4651: \emph{Cloud Computing and Big Data Systems} \hfill Spring \& Fall, 2017
%     \item[$\diamondsuit$] COMP4621: \emph{Computer Communication Networks I} \hfill Summer 2016
%     \item[$\diamondsuit$] COMP6611B: \emph{Topics on Cloud Computing and Data Analytics Systems} \hfill Fall 2015, 2016
% \end{innerlist}

% \vspace{4mm}
% % \vspace{-8mm}
% University of Toronto (Teaching Assistant)
% \vspace{2mm}
% \begin{innerlist}\itemsep0.25em
% \item[$\diamondsuit$] APS105 \emph{Computer Fundamentals} \hfill Fall 2010 -- 2014
% \item[$\diamondsuit$] ECE 344S \emph{Operating Systems} \hfill Spring 2011, 2012
% \item[$\diamondsuit$] ECE297 \emph{Communication and Design} \hfill Spring 2013, 2014
% \end{innerlist}

% \section{Advising and Supervisorship}
% \textbf{Current Students in the Doctor of Philosophy Program}\\[-1em]

% \newlength{\shortcollength}\setlength{\shortcollength}{1.85in}%
% \begin{tabular}[t]{@{}p{\textwidth-\shortcollength}p{\shortcollength}}
%     \emph{Name} & \emph{Duration of Study}\\
%     Minchen Yu & 2018 -- Present\\
%     Huangshi Tian & 2017 -- Present\\
%     Qizhen Weng & 2017 -- Present\\
%     Mingzhe Li (HKUST-SUSTech Joint PhD Program) & 2016 -- Present\\
%     Da Yan & 2016 -- Present\\
%     Chengliang Zhang & 2016 -- Present\\
%     Yinghao Yu (Co-supervised with Prof. Khaled B. Letaief) & 2015 -- Present\\
% \end{tabular}
% \vspace{4mm}

% % \newpage
% \textbf{Students Graduated with the Doctor of Philosophy Degree}\\
% \begin{tabular}[t]{@{}p{1.0in}p{2in}p{\textwidth-3.4in}}
%     \emph{Name} & \emph{Starting Date, Graduation Date and First Employment} &
%     \emph{Dissertation Title} \\
%     Chen Chen & 2014 -- 2018, Research Scientist at Huawei Research Lab, Hong Kong & Job Scheduling in the Cloud: A Tale on Fairness and Efficiency\\
% \end{tabular}
% \vspace{4mm}

% \textbf{Students Graduated with the Master of Philosophy Program}\\
% \begin{tabular}[t]{@{}p{1.0in}p{2in}p{\textwidth-3.4in}}
%     \emph{Name} & \emph{Starting Date, Graduation Date and First Employment} &
%     \emph{Dissertation Title} \\
%     Xiandong Qi & 2016 -- 2019, Software Engineer at AQUMON, Hong Kong. & Multi-Resource Fair Sharing with Constraints in Heterogeneous Clusters\\
% \end{tabular}
% \vspace{4mm}


% \vspace{4mm}
% \textbf{Student Supervision Achievements}\\[-0.6em]

% \emph{Hong Kong PhD Fellowship Scheme} (3-year fellowship for top PhD students in Hong Kong)\\[-0.7em]

% \begin{tabular}[t]{@{}p{\textwidth-\shortcollength}p{\shortcollength}}
%     \emph{Name of Awardee} & \emph{Duration of Fellowship}\\
%     Huangshi Tian & 2017 -- 2020\\
%     Qizhen Weng & 2017 -- 2020\\
%     Chengliang Zhang & 2016 -- 2019\\
%     Yinghao Yu & 2015 -- 2018
% \end{tabular}
% \vspace{4mm}

% \emph{Huawei PhD Fellowship Scheme} (4-year PhD fellowship for elite students sponsored by Huawei)\\[-0.7em]

% \begin{tabular}[t]{@{}p{\textwidth-\shortcollength}p{\shortcollength}}
%     \emph{Name of Awardee} & \emph{Duration of Fellowship}\\
%     Minchen Yu & 2018 -- 2022
% \end{tabular}
% \vspace{4mm}


% \emph{IEEE INFOCOM'17 Best-In-Session Presentation Award} \hfill May 2017 \\[-0.7em]

% \begin{innerlist}\itemsep0.25em
%   \item [$\diamondsuit$] Chen Chen, for his talk entitled ``Cluster Fair Queueing: Speeding up Data-Parallel Jobs with Delay Guarantees,'' in Conference Session \emph{Cloud Computing I}.

%   \item [$\diamondsuit$] Yinghao Yu, for his talk entitled ``LRC: Dependency-Aware Cache Management for Data Analytics Clusters,'' in Conference Session \emph{Big Data Processing}.
% \end{innerlist}

% \vspace{4mm}
% \textbf{Ph.D. Thesis Committees Served}\\[-0.7em]

% Ming Wen, 2019\\
% Li Chen, 2018\\
% Jingjie Jiang, Wei Bai, Yi Zhang, 2017\\
% Zhehui Wang, Zhice Yang, 2016

% \section{Professional Services}
% \textbf{Editorial Board}
% \vspace{2mm}
% \begin{innerlist}\itemsep0.25em
%     \item [$\diamondsuit$] Editor, China Communications \hfill 2017 -- Present
% \end{innerlist}
% \vspace{4mm}

% \textbf{Conference Organization}
% \vspace{2mm}
% \begin{innerlist}\itemsep0.25em
%     \item[$\diamondsuit$] Finance Chair, the 1st ACM Asia-Pacific Workshop on Networking (APNet 2017)
%     \item[$\diamondsuit$] Finance Chair, the 25th IEEE International Conference on Network Protocols (ICNP 2017)
%     \item[$\diamondsuit$] Student Travel Grant Chair, the 7th ACM SIGOPS Asia-Pacific Workshop on Systems (APSys 2016)
%     \item[$\diamondsuit$] Student organizer, IEEE INFOCOM 2014
%     \item[$\diamondsuit$] Student organizer, IEEE INFOCOM 2014 TPC Meeting
%     \item[$\diamondsuit$] Student organizer, the 21st IEEE/ACM International Symposium on Quality of Service (IWQoS 2013)
%     \item[$\diamondsuit$] Student organizer, the 22nd ACM Workshop on Network and Operating Systems Support for Digital Audio and Video (NOSSDAV 2012) 
% \end{innerlist}
% \vspace{4mm}

% \textbf{Technical Program Committee}
% \vspace{2mm}
% \begin{innerlist}\itemsep0.25em
%     \item[$\diamondsuit$] IEEE INFOCOM 2020
%     \item[$\diamondsuit$] IEEE GLOBECOM 2019: Next-Generation Networking and Internet
%     \item[$\diamondsuit$] The 39th IEEE International Conference on Distributed Computing Systems (ICDCS 2019)
%     \item[$\diamondsuit$] IEEE INFOCOM 2019
%     \item[$\diamondsuit$] The 26th IEEE International Conference on Network Protocols (ICNP 2018)
%     \item[$\diamondsuit$] The 28th ACM SIGMM Workshop on Network and Operating Systems Support for Digital Audio and Video (NOSSDAV 2018)
%     \item[$\diamondsuit$] IEEE GLOBECOM 2018: Big Data Track
%     \item[$\diamondsuit$] IEEE INFOCOM 2018
%     \item[$\diamondsuit$] IEEE ICNP 2017 Workshop on Machine Learning and Artificial Intelligence in Computer Networks (ML\&AI @ Network 2017)
%     \item[$\diamondsuit$] The 25th IEEE International Conference on Network Protocols (ICNP 2017) 
%     \item[$\diamondsuit$] IEEE GLOBECOM 2017: Next-Generation Networking and Internet Symposium
%     \item[$\diamondsuit$] The 27th ACM SIGMM Workshop on Network and Operating Systems Support for Digital Audio and Video (NOSSDAV 2017)
%     \item[$\diamondsuit$] The 26th International Conference on Computer Communication and Networks (ICCCN 2017)
%     \item[$\diamondsuit$] IEEE International Conference on Cloud Engineering (IC2E) 2017
%     \item[$\diamondsuit$] IEEE ICNP Workshop on Machine Learning in Computer Networks (NetworkML) 2016
%     \item[$\diamondsuit$] The 25th International Conference on Computer Communication and Networks (ICCCN 2016), Hot Topics in Networking Track (HOT)
%     \item[$\diamondsuit$] IEEE GLOBECOM 2015: Communications Software Services and Multimedia Applications (CSSMA) Symposium
% \end{innerlist}

% \vspace{4mm}

% \textbf{Reviewer for Journal Manuscript Submissions}
% \vspace{2mm}
% \begin{innerlist}\itemsep0.25em
%     \item[$\diamondsuit$] IEEE/ACM Transactions on Networking
%     \item[$\diamondsuit$] IEEE Transactions on Parallel and Distributed Systems
%     \item[$\diamondsuit$] IEEE Journal on Selected Areas in Communication 
%     \item[$\diamondsuit$] IEEE Transactions on Cloud Computing
%     \item[$\diamondsuit$] IEEE Transactions on Mobile Computing
%     \item[$\diamondsuit$] IEEE Transactions on Knowledge and Data Engineering
%     \item[$\diamondsuit$] IEEE Transactions on Information Forensics \& Security
%     \item[$\diamondsuit$] IEEE Transactions on Big Data
%     \item[$\diamondsuit$] IEEE Transactions on Services Computing
%     \item[$\diamondsuit$] IEEE Communications Letters
%     \item[$\diamondsuit$] Springer Multimedia Systems
%     \item[$\diamondsuit$] Springer Journal of Cloud Computing: Advances, Systems and Applications
%     \item[$\diamondsuit$] Springer Peer-to-Peer Networking and Applications
% \end{innerlist}

% \vspace{4mm}

% \textbf{Reviewer for Conference Manuscript Submissions}
% \vspace{2mm}
% \begin{innerlist}
%     \item[]
% IEEE INFOCOM,
% IEEE ICNP,
% IEEE ICDCS,
% IEEE GLOBECOM.
% \end{innerlist}



\section{Skills} 
\vspace{2mm}

\NumTabs{3}
\begin{inparaenum}[$\diamondsuit$]
\item Python
\tab\item Java, Scala
\tab\item Keras
\tab\item TensorFlow
\tab\item MXNet
\tab\item Spark
\tab\item Hadoop   
\end{inparaenum}


\section{References} 
\textbf{Wei Wang}, Assistant Professor \\
Hong Kong University of Science and Technology \\
Room 3524, CSE Department, HKUST \\
Clear Water Bay, Kowloon, Hong Kong \\ 
\textit{Phone:} +852--2358--6972 \\
\textit{Email:} \href{mailto:weiwa@cse.ust.hk}{weiwa@cse.ust.hk} \\
\textit{Web:} \url{https://www.cse.ust.hk/~weiwa/} \\ 

\textbf{Feng Yan}, Assistant Professor \\
University of Nevada, Reno \\
SEM 233, CSE Department
Reno, NV 89557\\ 
\textit{Phone:} +1--775--784--6448 \\
\textit{Email:} \href{mailto:fyan@unr.edu}{fyan@unr.edu} \\
\textit{Web:} \url{https://wolfweb.unr.edu/~fyan/} \\ 

\end{document}

%%%%%%%%%%%%%%%%%%%%%%%%%% End CV Document %%%%%%%%%%%%%%%%%%%%%%%%%%%%%

